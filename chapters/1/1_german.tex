\chapter{Teil 1: Charakter Erstellung}

\section{Kapitel 1: Schritt für Schritt}
Der erste Schritt, um einen Abenteurer im Dungeons & Dragons-Spiel zu spielen, besteht darin, sich einen eigenen Charakter auszudenken und zu erstellen. Dein Charakter ist eine Kombination aus Spielstatistiken, Rollenspielen und deiner Vorstellungskraft. Du wählst eine Rasse (Mensch oder Halbling) und eine Klasse (Kämpfer oder Zauberer). Du erfindest auch die Persönlichkeit, das Aussehen und die Hintergrundgeschichte deines Charakters. Sobald er fertig ist, dient dein Charakter als eine art Vertreter im Spiel, bzw. dein Avatar in der Dungeons and Dragons-Welt.\\
Bevor du unten mit Schritt 1 anfängst, überlege dir, welche Art von Abenteurer du spielen möchtest. Du könntest ein mutiger Kämpfer sein, ein hinterhältiger Schurke, ein leidenschaftlicher Kleriker oder ein extravaganter Zauberer. Oder du interessierst dich vielleicht mehr für einen unkonventionellen Charakter, wie einen muskulösen Schurken, der gut im Nahkampf ist, oder einen Bogenschützen, der Gegner aus der Ferne abschießt. Stehst du auf Fantasy-Fiction mit Zwergen oder Elfen? Falls ja, dann versuche einen Charakter mit einer dieser Rassen zu erstellen. Willst du, dass dein Charakter der härteste Abenteurer am Tisch ist? Dann werf am besten mal einen Blick in die Jagdklasse. Wenn du nicht weißt, wo du sonst anfangen sollst, schau dir die Illustrationen in diesem Buch an, um zu sehen, was dich interessiert.\\
Wenn Du einen Charakter im Kopf hast, führe einfach diese Schritte in der richtigen Reihenfolge aus und erstelle ihn so, dass die Werte den gewünschten Charakter widerspiegeln. Deine Vorstellung von deinem Charakter kann sich mit jeder Entscheidung, die du bei der Erstellung triffst, weiterentwickeln. Wichtig ist, dass du mit einem Charakter an den Tisch kommst, den du gerne spielst. In diesem Kapitel verwenden wir den Begriff \textbf{Character-Sheet}, um das einen Namen zu geben, was du zum Beschreiben deines Charakters verwendest, unabhängig davon, ob es sich um ein ofizielles Character-Sheet (wie zB die Bögen die du hier in diesem Werk findest), um eine Art digitaler Aufzeichnung oder um ein Notizbuch handelt. Ein offizielles DnD-Character-Sheet ist ein guter Ausgangspunkt, bis du weißt, welche Informationen du benötigst und wie du sie während des Spiels verwendest.
\subsubsection{Wir bauen Bruenor}
Jeder Schritt der Charaktererstellung enthält ein Beispiel, wir werden dir das an dem Beispiel von Spieler Bob zeigen der seinen Zwergencharakter \textit{Bruenor} erstellt.


\subsection{1. Wähle eine Rasse}
Jeder Charakter gehört zu einer Rasse, einer der vielen humanoiden Spezien in der DnD-Welt. Die häufigsten Spielercharakteren sind Zwerge, Elfen, Halblinge und Menschen. Einige Rassen haben auch Unterrasen wie Bergzwerg oder Waldelf. In Kapitel 2 findest du weitere Informationen zu diesen Rassen.\\
Die von dir gewählte Rasse trägt auf eine wichtige Weise zur Identität deines Charakters bei, indem er ein allgemeines Erscheinungsbild und die natürlichen Begabungen, die durch Kultur und Herkunft gewonnen wurden, festlegt. Die Rasse deines Charakters verleiht ihm bestimmten Eigenschaften, wie besonders ausgeprägte Sinne, bestimmte Waffen oder Werkzeuge, einer oder mehrere Fertigkeiten oder die Fähigkeit zum verwendung kleinerer Zauber. Diese Merkmale passen manchmal zu den Fähigkeiten bestimmter Klassen (siehe Schritt 2). Zum Beispiel machen die Rassen-Merkmale von Lichtfuß-Halblingen sie zu außergewöhnlichen Schurken, und Hochelfen sind in der Regel mächtige Zauberer. Manchmal macht es auch Spaß, gegen den Typ zu spielen. Halbling-Paladine und Bergzwergzauber können zum Beispiel ungewöhnliche, aber einprägsame Charaktere sein.\\
Deine Rasse erhöht außerdem einen oder mehrere deiner Fähigkeitswerte, die du in Schritt 3 festlegst. Beachte diesen Bonus und denke daran, sie später anzuwenden. Zeichne die Eigenschaften deiner Rasse auf deinem Charakcer-Sheet auf. Notiere dir gut auch die Startsprachen und die Basisgeschwindigkeit.

\subsubsection{Bruenor erstellen, Schritt 1}
Bob ist gerade dabei seinen Charakter zu kreieren. Er beschließt, dass ein schroffer Bergzwerg zu seiner Figur passt, die er spielen möchte. Auf seinem Charakcer-Sheet vermerkt er alle Rassenmerkmale der Zwerge, einschließlich seiner Geschwindigkeit von 25 Fuß und der Sprachen, die er kennt: Die in DnD übliche Sprache (Common) und Zwergisch.

\section{2. Wähle eine Klasse}
Jeder Abenteurer hat einer Klasse. Die Klasse beschreibt allgemein die Berufung eines Charakters, welche besonderen Talente er oder sie besitzt, und die Taktik, die er oder sie am wahrscheinlichsten anwendet, wenn er einen Dungeon erkundet, Monster bekämpft oder sich in einer angespannten Verhandlung befindet. Die Zeichenklassen werden in Kapitel 3 beschrieben.\\
Dein Charakter erhält eine Reihe von Vorteilen durch die Klassenwahl. Viele dieser Vorteile sind \textbf{Klassenfeatures} - Kompetenzen (einschließlich der Zaubersprüche), durch die sich dein Charakter von Mitgliedern anderer Klassen unterscheidet. Sie gewinnen dadurch auch eine Anzahl von \textbf{Fertigkeiten}: Rüstung, Waffen, Fähigkeiten, Rettungswürfe und manchmal Werkzeuge. Ihre Fähigkeiten definieren viele der Dinge, die Ihr Charakter besonders gut kann, von der Verwendung bestimmter Waffen bis hin zu einer überzeugendem Lügen.\\
Notiere auf Ihrem Charakterblatt alle Merkmale, die deine Klasse auf der 1. Stufe bietet.
\subsubsection{Level}

Normalerweise beginnt ein Charakter auf der ersten Ebene und steigt durch Abenteuern und \textbf{Erfahrungspunkte} (XP) an. Ein Charakter der ersten Stufe ist in der abenteuerlichen Welt unerfahren, obwohl er oder sie möglicherweise ein Soldat oder ein Pirat gewesen ist und in der Vergangenheitsgeschichte viele Abenteuer erlebt hat.\\
Beginne auf Level 1. wenn dein dein Charakter in das abenteuerliche Leben startet. Wenn du dich bereits mit dem Spiel auskennst oder einer bestehenden DnD-Kampagne beitretest, könnte dein DM entscheiden, dass Sie auf einer höheren Stufe beginnen, vorausgesetzt, Ihr Charakter hat bereits einige schwierige Abenteuer überstanden.\\
Schreibe dein Level auf deinem Character-Sheet auf. Wenn du auf einer höheren Stufe beginnst, notiere auch die zusätzlichen Punkte uä. die dir deine Klasse für dein Level nach dem 1. Level gibt. Notiere dir auch die Erfahrungspunkte. Ein Level 1 Charakter hat 0 XP. Ein übergeordneter Charakter beginnt in der Regel mit der Mindestanzahl an XP, die erforderlich ist, um dieses Level zu erreichen (siehe „Jenseits des ersten Levels“ weiter unten in diesem Kapitel).
\subsubsection{Hit Points und Hit Dice \\(Hit-Würfel)}
Die Trefferpunkte (Hit-Points) deines Charakters bestimmen, wie stark sich dein Charakter sich im Kampf und in anderen gefährlichen Situationen durchsetzen kann. Ihre Trefferpunkte werden von deinem Hit Dice (kurz für Hit Point Dice | Trefferpunkts-Würfel) bestimmt.\\
Auf Level 1 hat dein Charakter einen Hit-Dice, und der Würfeltyp wird von deiner Klasse bestimmt.
Du startest mir Trefferpunkten, die gleich dem höchsten Wurf dieses Würfels sind, wie in deiner Klassenbeschreibung angegeben. (Füge auch den Grundmodifikator hinzu, den wir in Schritt 3 Festlegen werden). Dies ist auch das \textbf{Maximum deiner Trefferpunkte}.\\
Notiere die Trefferpunkte deines Charakters auf deinem Character-Sheet. Schreibe auch den Typ des Hit-Dices auf, den dein Charakter verwendet, und die Anzahl der Hit-Dices, die du verwenden darfst. Wenn du sich ausruhst, kannst du Hit Dices eintauschen, um die Trefferpunkte wiederherzustellen (siehe „Ausruhen“ in Kapitel 8).

\subsubsection{Fertigkeitsbonus}
Die Tabelle, die in Ihrer Klassenbeschreibung angezeigt wird, zeigt Ihren Fertigkeitsbonus an, der für einen Charakter der ersten Stufe +2 ist. Dein Fähigkeitsbonus gilt für viele der Werte, die du auf Ihrem Character-Sheet aufzeichnen:
\begin{itemize}
	\item Hit-würfe mit Waffen, mit denen du umgehen kannst
	\item Hit-Würfe mit Zaubern, die du beherrschst
	\item Fähigkeitsüberprüfungen mit Fähigkeiten, die Sie beherrschen
	\item Fähigkeitsüberprüfungen mit Tools/Items, mit denen du dich auskennst
	\todo[inline]{Translate the following two Items correctly}
	\item Speichern von Würfen, die Sie beherrschen
	\item Speichern von Wurf-DCs für Zaubersprüche (erklärt in jeder Zauberspruchklasse)
\end{itemize}

Deine Klasse bestimmt Ihre Waffenfertigkeiten, \"saving throw proficiencies\"
\todo[inline]{Translate correclty}
und einige deiner Fähigkeiten und Werkzeug Fähigkeiten. (Fähigkeiten werden in Kapitel 7 beschrieben, Werkzeuge in Kapitel 5.) Deine Hintergrundgeschichte vermittelt dir zusätzliche Fähigkeiten und Werkzeug-Fertigkeiten. Natürlich bieten dir einige Rassen mehr Fertigkeiten als anderen. Vergewissere dich also, dass du alle diese Fähigkeiten sowie deinen  Bonus auf deinem Charakter-Sheet notierst.\\
Ihr Fertigkeitsbonus kann nicht mehr als einmal zu einem einzelnen Würfelwurf oder einer anderen Zahl hinzugefügt werden. Gelegentlich kann Ihr Fertigkeitsbonus geändert, zB verdoppelt oder halbiert werden, bevor du ihn anwendest. Wenn ein Umstand darauf hindeutet, dass Ihr Fertigkeitsbonus mehr als einmal für denselben Wurf gilt oder wenn er mehrmals multipliziert werden solle, füge ihn dennoch nur einmal hinzu, multiplizieren Sie ihn nur einmal und halbiere es nur einmal
\subsubsection{Bruenor erstellen, Schritt 2}
Bob stellt sich vor, dass Bruenor mit einer Axt in die Schlacht stürmt und ein Horn an seinem Helm ist abgebrochen. Er macht Bruenor zu einem Kämpfer und stellt die Fertigkeiten und Fähigkeiten der ersten Klasse auf seinem Charakter fest. %//
\todo[inline]{Translate correclty}

Als Level 1 Kämpfer hat Bruenor einen Hit Dice - einen d10 - und beginnt damit dTrefferpunkten, die 10 + seinem Grundmodifikator entsprechen. Bob merkt dies und wird die endgültige Nummer aufzeichnen, nachdem er die Verfassungsbewertung von Bruenor festgelegt hat (siehe Schritt 3). Bob notiert auch den Fähigkeitsbonus für einen 1st-Level-Charakter, der +2 ist.

\subsection{Ermitteln der Fähigkeitswerte}
