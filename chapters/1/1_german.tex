\chapter{Teil 1 - Dein Abenteurer}
\section{Kapitel 1: Charakter Erstellung Schritt für Schritt}
Der erste Schritt, um einen Abenteurer im Dungeons and Dragons-Spiel zu spielen, besteht darin, sich einen eigenen Charakter auszudenken und zu erstellen. Dein Charakter ist eine Kombination aus Spielstatistiken, Rollenspielen und deiner Vorstellungskraft. Du wählst ein Volk (Mensch oder Halbling) und eine Klasse (Kämpfer oder Zauberer). Du erfindest auch die Persönlichkeit, das Aussehen und die Hintergrundgeschichte deines Charakters. Sobald er fertig ist, dient dein Charakter als eine art Vertreter im Spiel, bzw. dein Avatar in der Dungeons and Dragons-Welt.\\
Bevor du unten mit Schritt 1 anfängst, überlege dir, welche Art von Abenteurer du spielen möchtest. Du könntest ein mutiger Kämpfer sein, ein hinterhältiger Schurke, ein leidenschaftlicher Kleriker oder ein extravaganter Zauberer. Oder du interessierst dich vielleicht mehr für einen unkonventionellen Charakter, wie einen muskulösen Schurken, der gut im Nahkampf ist, oder einen Bogenschützen, der Gegner aus der Ferne abschießt. Stehst du auf Fantasy-Fiction mit Zwergen oder Elfen? Falls ja, dann versuche einen Charakter mit einer dieser Völker zu erstellen. Willst du, dass dein Charakter der härteste Abenteurer am Tisch ist? Dann werf am besten mal einen Blick in die Jagdklasse. Wenn du nicht weißt, wo du sonst anfangen sollst, schau dir die Illustrationen in diesem Buch an, um zu sehen, was dich interessiert.\\
Wenn Du einen Charakter im Kopf hast, führe einfach diese Schritte in der richtigen Reihenfolge aus und erstelle ihn so, dass die Werte den gewünschten Charakter widerspiegeln. Deine Vorstellung von deinem Charakter kann sich mit jeder Entscheidung, die du bei der Erstellung triffst, weiterentwickeln. Wichtig ist, dass du mit einem Charakter an den Tisch kommst, den du gerne spielst. In diesem Kapitel verwenden wir den Begriff \textbf{Character-Sheet}, um das einen Namen zu geben, was du zum Beschreiben deines Charakters verwendest, unabhängig davon, ob es sich um ein ofizielles Character-Sheet (wie zB die Bögen die du hier in diesem Werk findest), um eine Art digitaler Aufzeichnung oder um ein Notizbuch handelt. Ein offizielles DnD-Character-Sheet ist ein guter Ausgangspunkt, bis du weißt, welche Informationen du benötigst und wie du sie während des Spiels verwendest.
\subsubsection{Wir erstellen Bruenor}
Jeder Schritt der Charaktererstellung enthält ein Beispiel, wir werden dir das an dem Beispiel von Spieler Bob zeigen der seinen Zwergencharakter \textit{Bruenor} erstellt.

\subsection*{1. Wähle ein Volk}
Jeder Charakter gehört zu einem Volk, einer der vielen humanoiden Spezien in der DnD-Welt. Die häufigsten Spielercharakteren sind Zwerge, Elfen, Halblinge und Menschen. Einige Völker haben auch Untervölker wie Bergzwerg oder Waldelf. In Kapitel 2 findest du weitere Informationen zu diesen Völkern.\\
Das von dir gewählte Volk trägt auf eine wichtige Weise zur Identität deines Charakters bei, indem er ein allgemeines Erscheinungsbild und die natürlichen Begabungen, die durch Kultur und Herkunft gewonnen wurden, festlegt. Das Volk deines Charakters verleiht ihm bestimmten Eigenschaften, wie besonders ausgeprägte Sinne, bestimmte Waffen oder Werkzeuge, einer oder mehrere Fertigkeiten oder die Fähigkeit zum verwendung kleinerer Zauber. Diese Merkmale passen manchmal zu den Fähigkeiten bestimmter Klassen (siehe Schritt 2). Zum Beispiel machen die Völker-Merkmale von Lichtfuß-Halblingen sie zu außergewöhnlichen Schurken, und Hochelfen sind in der Regel mächtige Zauberer. Manchmal macht es auch Spaß, gegen den Typ zu spielen. Halbling-Paladine und Bergzwergzauber können zum Beispiel ungewöhnliche, aber einprägsame Charaktere sein.\\
Dein Volk erhöht außerdem einen oder mehrere deiner Fähigkeitswerte, die du in Schritt 3 festlegst. Beachte diesen Bonus und denke daran, sie später anzuwenden. Zeichne die Eigenschaften deines Volkes auf deinem Charakcer-Sheet auf. Notiere dir gut auch die Startsprachen und die Basisgeschwindigkeit.

\subsubsection{Bruenor erstellen, Schritt 1}
Bob ist gerade dabei seinen Charakter zu kreieren. Er beschließt, dass ein schroffer Bergzwerg zu seiner Figur passt, die er spielen möchte. Auf seinem Charakcer-Sheet vermerkt er alle Völkermerkmale der Zwerge, einschließlich seiner Geschwindigkeit von 25 Fuß und der Sprachen, die er kennt: Die in DnD übliche Sprache (Common) und Zwergisch.

\subsection{2. Wähle eine Klasse}
Jeder Abenteurer hat einer Klasse. Die Klasse beschreibt allgemein die Berufung eines Charakters, welche besonderen Talente er oder sie besitzt, und die Taktik, die er oder sie am wahrscheinlichsten anwendet, wenn er einen Dungeon erkundet, Monster bekämpft oder sich in einer angespannten Verhandlung befindet. Die Zeichenklassen werden in Kapitel 3 beschrieben.\\
Dein Charakter erhält eine Reihe von Vorteilen durch die Klassenwahl. Viele dieser Vorteile sind \textbf{Klassenfeatures} - Kompetenzen (einschließlich der Zaubersprüche), durch die sich dein Charakter von Mitgliedern anderer Klassen unterscheidet. Sie gewinnen dadurch auch eine Anzahl von \textbf{Fertigkeiten}: Rüstung, Waffen, Fähigkeiten, Rettungswürfe und manchmal Werkzeuge. Ihre Fähigkeiten definieren viele der Dinge, die Ihr Charakter besonders gut kann, von der Verwendung bestimmter Waffen bis hin zu einer überzeugendem Lügen.\\
Notiere auf Ihrem Charakterblatt alle Merkmale, die deine Klasse auf der 1. Stufe bietet.

\subsubsection{Level}
Normalerweise beginnt ein Charakter auf der ersten Ebene und steigt durch Abenteuern und \textbf{Erfahrungspunkte} (XP) an. Ein Charakter der ersten Stufe ist in der abenteuerlichen Welt unerfahren, obwohl er oder sie möglicherweise ein Soldat oder ein Pirat gewesen ist und in der Vergangenheitsgeschichte viele Abenteuer erlebt hat.\\
Beginne auf Level 1. wenn dein dein Charakter in das abenteuerliche Leben startet. Wenn du dich bereits mit dem Spiel auskennst oder einer bestehenden DnD-Kampagne beitretest, könnte dein DM entscheiden, dass Sie auf einer höheren Stufe beginnen, vorausgesetzt, Ihr Charakter hat bereits einige schwierige Abenteuer überstanden.\\
Schreibe dein Level auf deinem Character-Sheet auf. Wenn du auf einer höheren Stufe beginnst, notiere auch die zusätzlichen Punkte uä. die dir deine Klasse für dein Level nach dem 1. Level gibt. Notiere dir auch die Erfahrungspunkte. Ein Level 1 Charakter hat 0 XP. Ein übergeordneter Charakter beginnt in der Regel mit der Mindestanzahl an XP, die erforderlich ist, um dieses Level zu erreichen (siehe „Jenseits des ersten Levels“ weiter unten in diesem Kapitel).
\subsubsection{Trefferpunkte und Hit Dice (Hit-Würfel)}
Die Trefferpunkte (Hit-Points) deines Charakters bestimmen, wie stark sich dein Charakter sich im Kampf und in anderen gefährlichen Situationen durchsetzen kann. Ihre Trefferpunkte werden von deinem Hit Dice (kurz für Hit Point Dice | Trefferpunkts-Würfel) bestimmt.\\
Auf Level 1 hat dein Charakter einen Hit-Dice, und der Würfeltyp wird von deiner Klasse bestimmt.
Du startest mir Trefferpunkten, die gleich dem höchsten Wurf dieses Würfels sind, wie in deiner Klassenbeschreibung angegeben. (Füge auch den Grundmodifikator hinzu, den wir in Schritt 3 Festlegen werden). Dies ist auch das \textbf{Maximum deiner Trefferpunkte}.\\
Notiere die Trefferpunkte deines Charakters auf deinem Character-Sheet. Schreibe auch den Typ des Hit-Dices auf, den dein Charakter verwendet, und die Anzahl der Hit-Dices, die du verwenden darfst. Wenn du sich ausruhst, kannst du Hit-Dices eintauschen, um die Trefferpunkte wiederherzustellen (siehe „Ausruhen“ in Kapitel 8).

\subsubsection{Fertigkeitsbonus}
Die Tabelle, die in Ihrer Klassenbeschreibung angezeigt wird, zeigt Ihren Fertigkeitsbonus an, der für einen Charakter der ersten Stufe +2 ist. Dein Fähigkeitsbonus gilt für viele der Werte, die du auf Ihrem Character-Sheet aufzeichnen:
\begin{itemize}
	\item Hit-würfe mit Waffen, mit denen du umgehen kannst
	\item Hit-Würfe mit Zaubern, die du beherrschst
	\item Fähigkeitsüberprüfungen mit Fähigkeiten, die Sie beherrschen
	\item Fähigkeitsüberprüfungen mit Tools/Items, mit denen du dich auskennst
	\todo[inline]{Translate the following two Items correctly}
	\item Speichern von Würfen, die Sie beherrschen
	\item Speichern von Wurf-DCs für Zaubersprüche (erklärt in jeder Zauberspruchklasse)
\end{itemize}

Deine Klasse bestimmt Ihre Waffenfertigkeiten, Rettungswürfe und einige deiner Fähigkeiten und Werkzeug Fähigkeiten. (Fähigkeiten werden in Kapitel 7 beschrieben, Werkzeuge in Kapitel 5.) Deine Hintergrundgeschichte vermittelt dir zusätzliche Fähigkeiten und Werkzeug-Fertigkeiten. Natürlich bieten dir einige Völker mehr Fertigkeiten als anderen. Vergewissere dich also, dass du alle diese Fähigkeiten sowie deinen  Bonus auf deinem Charakter-Sheet notierst.\\
Ihr Fertigkeitsbonus kann nicht mehr als einmal zu einem einzelnen Würfelwurf oder einer anderen Zahl hinzugefügt werden. Gelegentlich kann Ihr Fertigkeitsbonus geändert, zB verdoppelt oder halbiert werden, bevor du ihn anwendest. Wenn ein Umstand darauf hindeutet, dass Ihr Fertigkeitsbonus mehr als einmal für denselben Wurf gilt oder wenn er mehrmals multipliziert werden solle, füge ihn dennoch nur einmal hinzu, multiplizieren Sie ihn nur einmal und halbiere es nur einmal
\subsubsection{Bruenor erstellen, Schritt 2}
Bob stellt sich vor, dass Bruenor mit einer Axt in die Schlacht stürmt und ein Horn an seinem Helm ist abgebrochen. Er macht Bruenor zu einem Kämpfer und stellt die Fertigkeiten und Fähigkeiten der ersten Klasse auf seinem Charakter fest.
\todo[inline]{Translate correclty}

Als Level 1 Kämpfer hat Bruenor einen Hit-Dice - einen d10 - und beginnt damit dTrefferpunkten, die 10 + seinem Grundmodifikator entsprechen. Bob merkt dies und wird die endgültige Nummer aufzeichnen, nachdem er die Verfassungsbewertung von Bruenor festgelegt hat (siehe Schritt 3). Bob notiert auch den Fähigkeitsbonus für einen 1st-Level-Charakter, der +2 ist.

\subsection{3. Ermitteln der Fähigkeitswerte}
Vieles, was dein Charakter im Spiel macht, hängt von seinen sechs Fähigkeiten ab:
\textbf{Stärke}, \textbf{Geschicklichkeit}, \textbf{Verfassung}, \textbf{Intelligenz}, \textbf{Weisheit} und \textbf{Charisma}. Jede Fähigkeit hat einen Wert den du	auf deinem Charakter-Sheet aufschreibst. \\
Die sechs Fähigkeiten und ihre Verwendung im Spiel werden in Kapitel 7 beschrieben. Die Tabelle mit den Fähigkeitswerten bietet einen schnellen Überblick darüber, welche Vorteile von jeder Fähigkeit berechnet werden, welche Völker welche Fähigkeiten erhöhen und welche Klassen jede Fähigkeit als besonders wichtig erachten.\\
Die 6 Fähigkeitswerte werden zufällig bestimmt. Wirf vier 6-seitige Würfel und notiere die Summe der höchsten drei Würfel auf einem Blatt Papier.
Mache dies 5 weitere Male, so dass du sechs Zahlen erwürfelt hast. Wenn du Zeit sparen möchtest, die Idee nicht magst oder generell immer ein Pechvogel bist kannst du stattdessen die folgenden Werte verwenden: 15, 14, 13, 12, 10, 8.\\
Nehme nun die sechs Zahlen und schreibe dir jede Zahl neben eine der sechs Fähigkeiten deines Charakters, um Stärke, Geschicklichkeit, Konstitution, Intelligenz, Weisheit und Charisma einen Wert zu bewerten. Nehme danach Änderungen an Ihren Fähigkeitswerten vor, die sich aus deiner Völkerwahl ergeben.\\
Bestimme nach dem Zuordnen deiner Fähigkeitswerte Ihre \textbf{Fähigkeitsmodifikatoren} mithilfe der Tabellen "Fähigkeitswerte" und "Modifikatoren". Um einen Fähigkeitsmodifikator zu ermitteln, ohne die Hilfe der Tabelle, subtrahiere 10 von dem Fähigkeitswert und dividiere das Ergebnis durch 2 (Abrunden). Schreibe dann den Modifikator neben jede deiner Werte.
\newpage
\begin{commentbox}{Schneller Aufbau}
Jede Klassenbeschreibung in Kapitel 3 enthält einen Abschnitt mit Vorschlägen zum schnellen Erstellen eines Charakters dieser Klasse. Dazu gehören das Zuordnen der höchsten Fähigkeitswerte, ein für die Klasse geeigneter Hintergrund und Startzauber.
\end{commentbox}
\header{Fähigkeiten und Modifikatoren}

\begin{minipage}{.45\linewidth}
	\begin{dndtable}
	   	\textbf{Punktzahl}  & \textbf{Modifikator} \\
			1 & -5 \\
			2-3 & -4 \\
			4-5 & -3 \\
			6-7 & -2 \\
			8-9 & -1 \\
			10-11 & +0 \\
			12-13 & +1 \\
			14-15 & +2 \\
	\end{dndtable}
\end{minipage}
\begin{minipage}{.45\linewidth}
	\begin{dndtable}
	   	\textbf{Punktzahl}  & \textbf{Modifikator} \\
			16-17 & +3 \\
			18-19 & +4 \\
			20-21 & +5 \\
			22-23 & +6 \\
			24-25 & +7 \\
			26-27 & +8 \\
			28-29 & +9 \\
			30 & +0 \\
	\end{dndtable}
\end{minipage}

\header{Fähigkeitspunkte Tabelle}
\begin{dndtable}
\\
\textbf{Stärke} \\
\textit{Eigenschaften:} \newline
\ \ \ \ Natürliche Sportlichkeit, körperliche Kraft \newline
\textit{Wichtig für:} \newline
\ \ \ \ Kämpfer \newline
\textit{Völkerbonus:} \newline
\ \ \ \ Bergzwerg (+2) \ \ Mensch (+1)

\end{dndtable}

\begin{dndtable}
	\textbf{Geschicklichkeit} \\
	\textit{Eigenschaften:} \newline
	\ \ \ \ Körperliche Beweglichkeit, Reflexe, Gleichgewicht, Ausgeglichenheit \newline
	\textit{Wichtig für:} \newline
	\ \ \ \ Schurke \newline
	\textit{Völkerbonus} \newline
	\ \ \ \ Elf (+2)  \ \ Halbling (+2) \newline
	\ \ \ \ Mensch (+1)
\end{dndtable}

\begin{dndtable}
	\textbf{Konstitution} \\
	\textit{Eigenschaften:} \newline
	\ \ \ \ Gesundheit, Ausdauer, Lebenskraft \newline
	\textit{Wichtig für:} \newline
	\ \ \ \ Jeden \newline
	\textit{Völkerbonus} \newline
	\ \ \ \ Zwerg (+2) \newline
	\ \ \ \ beleibter Halbling (+1) \newline
	\ \ \ \ Mensch (+1)
\end{dndtable}

\begin{dndtable}
	\textbf{Intelligenz} \\
	\textit{Eigenschaften:} \newline
	\ \ \ \ Geistesschärfe, Erinnern von Informationen, analytische Fähigkeiten \newline
	\textit{Wichtig für:} \newline
	\ \ \ \ Zauberer \newline
	\textit{Völkerbonus} \newline
	\ \ \ \ Großer Elf (+1) \newline
	\ \ \ \ Mensch (+1)
\end{dndtable}

\begin{dndtable}
	\textbf{Weisheit} \\
	\textit{Eigenschaften:} \newline
	\ \ \ \ Wahrnehmung, Intuition \newline
	\textit{Wichtig für:} \newline
	\ \ \ \ Kleriker \newline
	\textit{Völkerbonus} \newline
	\ \ \ \ Berg-Zwerg (+1) \newline
	\ \ \ \ Wald-Elf (+1) \newline
	\ \ \ \ Mensch (+1)
\end{dndtable}

\begin{dndtable}
	\textbf{Charisma} \\
	\textit{Eigenschaften:} \newline
	\ \ \ \ Vertrauen, Beredsamkeit, Anführung \newline
	\textit{Wichtig für:} \newline
	\ \ \ \ Anführende und diplomatische Figuren \newline
	\textit{Völkerbonus} \newline
	\ \ \ \ Leichtfüßiger Halbling (+1) \newline
	\ \ \ \ Mensch (+1)
\end{dndtable}
\subsubsection{Bruenor erstellen, Schritt 3}
Bob schreibt Bruenors finale Trefferpunkte auf:
10 + seinen Konstitutionsmodifikator (der +3 beträgt) also insgesamt 13 Trefferpunkte.

\subsubsection{Variante: individuelles anpassen der Fähigkeitspunkte}
Es kann passieren dass euer Dungeon-Master diese Variante bevorzugt. Bei dieser Variante kannst du deinem Charakter eine Anzahl an Fähigkeitswerten individuell verteilen die je nach ihrer Anzahl Punkte kosten. Dir stehen 27 Punkte zur verfügung. Die Kosten für jeden Fähigkeitswert werden in der unteren Tabelle angezeigt. Zum Beispiel kostet eine Punktzahl von 14 genau 7 Punkte. Mit dieser Methode ist 15 die höchste Fähigkeitsbewertung, die du erzielen kannst, bevor du deinen Völkerbonus anwendest. Du kannst aber auch nicht weniger als 8 Punkte erzielen.\\
Mit dieser Methode zur Ermittlung der Fähigkeitswerte kannst du einen Wertesatz von drei hohen und drei niedrigen (15, 15, 15, 8, 8, 8) erstellen, einen Wertesatz von Zahlen, die über dem Durchschnitt liegen und nahezu gleich sind (13, 13, 13, 12, 12, 12) oder eine beliebige Anzahl von Zahlen zwischen diesen Extremen.

\header{Punktekosten für Fähigkeit}
\begin{dndtable}
Fähigkeitswert & Kosten \\
8 & 0\\
9 & 1\\
10 & 2\\
11 & 3 \\
12 & 4 \\
13 & 5 \\
14 & 7 \\
15 & 9 \\
\end{dndtable}
\subsection{4. Beschreibe deinen Charakter}
Sobald du die grundlegenden Spielaspekte deines Charakters kennst, ist es an der Zeit, die Persönlichkeit dieser Person auszuarbeiten. Dein Charakter braucht einen Namen. Überlege dir eine Weile, wie er oder sie aussieht und wie er sich allgemein verhält. \\
Anhand der Informationen in Kapitel 4 kannst du das Aussehen und die Persönlichkeitsmerkmale deines Charakters genauer definieren. Schreibe die \textbf{Moral} deines Charakters und seine \textbf{Ideale} auf. Kapitel 4 hilft dir auch dabei, die Dinge herauszufinden, die deinem Charakter am wichtigsten sind, vielleicht einige \textbf{Grenzen}, und seine Macken, die ihn eines Tages untergraben könnten. \\
Der \textbf{Hintergrund} den Ursprung, die Hintergrundgeschichte, seine oder ihre ursprüngliche Tätigkeit und die Rolle deines Charakters in der DnD-Welt. Euer DM bietet euch möglicherweise zusätzliche Hintergründe zu den in Kapitel 4 enthaltenen Hintergründen an und ist möglicherweise bereit, gemeinsam mit dir einen eigenen Hintergrund zu entwickeln, der genauer zu deinem Charakterkonzept passt. Ein Hintergrund verleiht deinem Charakter eine Hintergrundfeature (einen allgemeinen Vorteil) und Kenntnisse in zwei Skills, und es kann dir zusätzliche Sprachen oder Skills für den Umgang bestimmten Arten von Werkzeugen geben. Notiere diese Informationen zusammen mit den von dir entwickelten Persönlichkeitsdaten in deinem Charakterblatt.
\subsubsection{Fähigkeiten}
Berücksichtige die Fähigkeitswerte deines Charakters, wenn du sein Aussehen und seine Persönlichkeit ausarbeitest. Ein sehr starker Charakter mit niedriger Intelligenz kann anders denken als ein sehr intelligenter Charakter mit geringer Stärke.\\
Zum Beispiel entspricht hohe Stärke normalerweise einem stämmigen oder athletischen Körper, während ein Charakter mit niedriger Stärke dürr oder prall sein kann.\\
Ein Charakter mit hoher Geschicklichkeit ist wahrscheinlich geschmeidig und schlank, während ein Charakter mit geringer Geschicklichkeit entweder unberechenbar und unbeholfen oder schwer und aufgrund seiner dicken Fingern nicht mit kleinen Sachen interagieren kann.\\
Ein Charakter mit hoher Konstitution sieht normalerweise gesund aus, mit hellen Augen und reichlich Energie. Ein Charakter mit niedriger Konstitution kann kränklich oder schwach sein.\\
Ein Charakter mit hoher Intelligenz ist möglicherweise sehr neugierig und fleißig, während ein Charakter mit niedriger Intelligenz möglicherweise nur einfache Worte spricht oder einfache Sachen direkt wieder vergisst.\\
Ein Charakter mit hoher Weisheit hat ein gutes Urteilsvermögen, Einfühlungsvermögen und ein allgemein gutes Bewusstsein darüber, wie eine Situation gerade abläuft. Ein Charakter mit niedriger Weisheit könnte abwesend, töricht oder unbesonnen sein.\\
Ein Charakter mit hohem Charisma strahlt Vertrauen aus, das normalerweise mit einer anmutigen oder einschüchternden Präsenz vermischt wird. Ein Charakter mit niedrigem Charisma kann als abrasiv, unartikuliert oder schüchtern wirken.

\subsubsection{Bruenor erstellen, Schritt 4}
Bob gibt einige grundlegende Details von Bruenor an: seinen Namen, sein Geschlecht (männlich), seine Größe und sein Gewicht sowie seine Moral (hält sich immer an Regeln). Seine hohe Stärke und Konstitution lassen auf einen gesunden, athletischen Körper schließen, und seine geringe Intelligenz lässt auf ein gewisses Maß an Vergesslichkeit schließen.\\
Bob beschließt, dass Bruenor aus einer adeligen Familienlinie stammt. Jedoch wurde sein Clan wurde aus seiner Heimat vertrieben, als Bruenor noch sehr jung war. Er wuchs als Schmied in den entlegenen Dörfern von Icewind Dale auf. Aber Bruenor hat ein heroisches Schicksal - um seine Heimat zurückzugewinnen -, also wählt Bob den Hintergrund des Volkshelden für seinen Zwerg. Er stellt fest, welche Fähigkeiten und Besonderheiten dieser Hintergrund ihm bietet.\\
Bob hat ein ziemlich klares Bild von Bruenors Persönlichkeit. Er lässt die im Volkshelden-Hintergrund vorgeschlagenen Persönlichkeitsmerkmale außer Acht und bemerkt, dass Bruenor ein fürsorglicher, sensibler Zwerg ist, der seine Freunde und Verbündeten wirklich liebt, aber er versteckt diesen weichen Kern hinter einer harten Schale. Er wählt das Ideal der Fairness aus der Liste in seinem Hintergrund aus und bemerkt, dass Bruenor glaubt, dass niemand über dem Gesetz steht.\\
In Anbetracht seiner Geschichte ist Bruenors Leidenschaft offensichtlich: Er möchte Mithral Hall, sein Heimatland, eines Tages von dem Schattendrachen, der die Zwerge vertrieben hat, zurückfordern. Seine Schwäche ist an seine fürsorgliche, einfühlsame Art gebunden - er hat ein Herz für Waisenkinder und eigensinnige Seelen, was dazu führt, dass er selbst dann Erbarmen zeigt, wenn es nicht gerechtfertigt ist.\\

\subsection{5. Wähle Ausrüstung}
Deine Klasse und Dein Hintergrund bestimmen die Startausrüstung deines Charakters, einschließlich Waffen, Rüstung und anderer Ausrüstung für deinen Abenteurer. Schreibe dein Inventar auf dein Charakterblatt auf. Alle dieser Items sind in Kapitel 5, „Ausrüstung“, detailliert beschrieben.\\
Anstatt die Ausrüstung, die du von deiner Klasse und deinem Hintergrund erhalten hast mitzunehmen, kannst du deine Startausrüstung auch kaufen. Je nach deiner Klasse hast du eine bestimmte Anzahl an Goldstücken (\textbf{Goldpieces}/GP) die du wie du willst ausgeben kannst, aber mehr dazu in Kapitel 5. Umfangreiche Itemlisten mit Preisen findest du auch dort in diesem Kapitel. Wenn du möchtest, kannst du auch ein kleines Objekt mit wenig Wert kostenlos erhalten (siehe Tabelle „Trinkets“ am Ende von Kapitel 5).\\
Deine Stärkewert begrenzt die Anzahl der Ausrüstung, die du tragen kannst. Kaufe keine Ausrüstung mit einem Gesamtgewicht, das deine Stärke-Punktzahl mal 15 übersteigt. (Kompatibilitätshalber mit Modulen und anderen DnD Büchern und Regelwerken die nicht frei zugänglich oder nur in Englisch vorhanden sind rechnen wir in Pfund und nicht in kg). In Kapitel 7 findest du weitere Informationen zur Tragfähigkeit.

\subsubsection{Rüstungsklassen (AC)}
Deine Rüstungsklasse (\textbf{Armor Class} / AC) gibt an, wie stark dein Charakter im Kampf verletzt wird. Dinge, die zu deinem AC beitragen, umfassen die Rüstung, die du trägst, den Schild, den du trägst, und deinen Geschicklichkeitsmodifikator. Allerdings tragen nicht alle Charaktere Rüstungen oder Schilde.\\
Ohne Rüstung oder Schild entspricht die AC deines Charakters 10 + seinem oder ihrem Geschicklichkeitsmodifikator. Wenn dein Charakter Rüstung, einen Schild oder beides trägt, berechneest du AC nach den Regeln in Kapitel 5. Notieren dir deine AC in deinem Charakterblatt.\\
Dein Charakter muss mit Rüstungen und Schildern gut umgehen können, um sie effektiv zu tragen und zu nutzen, und deine Rüstungs- und Schildfähigkeiten werden von deiner Klasse bestimmt. Das Tragen einer Rüstung oder eines Schildes hat Nachteile, wenn dein Charakter nicht die erforderlichen Kenntnisse besitzt (Siehe Kapitel 5).\\
Einige Zaubersprüche und Klassenfeatures bieten dir eine andere Möglichkeit, deine AC zu berechnen. Wenn du über mehrere Funktionen verfügst, mit denen du deinen AC auf verschiedene Arten berechnen kannst, wähle die zu verwendende aus.\\

\subsubsection{Waffen}
Berechnen Sie für jede Waffe, die Ihr Charakter hat, den Modifikator, den Sie beim Angriff mit der Waffe verwenden, und den Schaden, den Sie beim Schlagen verursachen.\\
Wenn Sie mit einer Waffe angreifen, würfeln Sie einen d20 und addieren Ihren Leistungsbonus (aber nur, wenn Sie mit der Waffe vertraut sind) und den entsprechenden Fähigkeitsmodifikator.\\
\begin{itemize}
	\item Verwenden Sie für Angriffe mit \textbf{Nahkampfwaffen} Ihren Stärkemodifizierer für Angriffs- und Schadenswurf. Eine Waffe mit der Finesse-Eigenschaft, beispielsweise ein Greifer, kann stattdessen Ihren Geschicklichkeitsmodifikator verwenden.
	\item Verwenden Sie für Angriffe mit \textbf{Fernkampfwaffen} Ihren Geschicklichkeitsmodifikator für Angriffs- und Schadenswurf. Eine Waffe, die über die geworfene Eigenschaft verfügt, z. B. eine Handaxe, kann stattdessen Ihren Stärkemodifizierer verwenden.
\end{itemize}

\subsubsection{Bruenor erstellen, Schritt 5}
Bob notiert die Startausrüstung aus der Kampfklasse und dem Hintergrund des Volkshelden. Seine Startausrüstung umfasst Kettenhemd und ein Schild, die Bruenor eine Rüstungsklasse von 18 verleihen.\\
Für Bruenors Waffen wählt Bob eine Kampfaxt und zwei Handäxte. Seine Kampfaxt ist eine Nahkampfwaffe. Bruenor verwendet seinen Stärkemodifikator für seine Angriffe und seinen Schaden. Sein Angriffsbonus ist sein Stärkemodifizierer (+3) und sein Fähigkeitsbonus (+2) also insgesamt +5. Die Kampfaxt fügt 1W8 Schaden zu, und Bruenor fügt,  wenn er trifft, dem Schaden seinen Stärkemodifizierer hinzu, insgesamt also 1W8 + 3 Schaden. Wenn er eine Handaxt wirfst, hat Bruenor denselben Angriffsbonus (Handaxe, wie geworfene Waffen, verwende dafür Stärke für Angriffe und Schaden), und die Waffe fügt bei Treffer 1W6 + 3 Schaden zu.\\

\subsection{6. Versammelt euch ihr Helden!}
Die meisten DnD-Charaktere funktionieren nicht alleine. Jeder Charakter spielt eine Rolle in einem eine \textbf{Gruppe} von Abenteurern, die für einen gemeinsamen Zweck zusammenarbeiten. Teamwork und Zusammenarbeit verbessern die Chancen Ihrer Gruppe, die vielen Gefahren in der Welt von Dungeons and Dragons zu überleben. Sprich mit deinen Mitspielern und deinem DM, um zu entscheiden, ob deine Charaktere einander kennen, wie sie sich getroffen haben und welche Art von Quests die Gruppe unternehmen könnte.

\subsection{7. Jenseits der ersten Stufe!}
Wenn dein Charakter auf Abenteuerreise geht und Herausforderungen meistert, gewinnt er Erfahrung, die durch Erfahrungspunkte dargestellt wird. Ein Charakter, der einen bestimmten Erfahrungspunktwert erreicht, macht Fortschritte in seinen Fähigkeiten. Diese Entwicklung wird als das \textbf{Erreichen eines Levels} bezeichnet.\\

\subsection{Spielstufen}
Wenn dein Charakter ein Level erreicht, gewährt dir seine Klasse oft zusätzliche Features, die in der Klassenbeschreibung beschrieben stehen. Mit einigen dieser Funktionen kannst du deine Fähigkeitspunkte erhöhen, indem du entweder zwei Werte um 1 oder einen Wert um 2 erhöhst. Du kannst einen Fähigkeitswert nicht über 20 erhöhen. Zusätzlich erhöht sich der Fähigkeitsbonus jedes Charakters auf bestimmten Stufen.\\
Jedes Mal, wenn du ein Level erreichst, erhälst du einen zusätzlichen Hit Würfel. Wirf den Hit Würfel, füge den Konstitutionsmodifikator der Rolle hinzu und addiere die Summe zu deinem Trefferpunkt-Maximum. Alternativ kannst du den festen Wert verwenden, der in deiner Klassenbeschreibung angezeigt wird. Dies ist das durchschnittliche Ergebnis des Würfelwurfs (aufgerundet).\\
Wenn dein Konstitutionsmodifikator um 1 erhöht wird, erhöht sich Ihr Trefferpunktmaximum für jedes erreichte Niveau um 1. Wenn Bruenor beispielsweise als Kämpfer Level 8 erreicht, erhöht er seine Konstitution von 17 auf 18 und erhöht damit seinen Konstitutionsmodifikator von +3 auf +4. Sein Trefferpunktmaximum steigt dann um 8 an.\\
Die Charakterfortschrittstabelle fasst die XP zusammen, die du benötigst, um in den Leveln von Level 1 bis Level 20 aufzusteigen, und den Leistungsbonus für einen Charakter dieses Levels. Lese die Informationen in deiner Charakterklassem Beschreibung um zu sehen welche Verbesserungen du bei jedem Levelaufstieg noch dazu bekommst.

\subsubsection{Spielstufen}
Die Schattierung in der Charakterfortschrittstabelle zeigt die vier Spielebenen. Mit den Stufen sind keine Regeln verbunden. Sie beschreiben allgemein, wie sich das Spielerlebnis ändert, wenn die Charaktere an Bedeutung gewinnen.
In der ersten Reihe (Stufen 1–4) sind Charaktere effektiv Auszubildende. Sie lernen die Merkmale, die sie als Mitglieder bestimmter Klassen definieren,
einschließlich der wichtigsten Optionen, die ihre Klassenmerkmale mit sich bringen, wenn sie fortschreiten (wie z. B. die arkanische Tradition eines Zauberers oder ein Martial Archetype eines Kämpfers). Die Bedrohungen, mit denen sie konfrontiert sind, sind relativ gering und stellen in der Regel eine Gefahr für lokale Bauernhöfe oder Dörfer dar.
In der zweiten Ebene (Level 5–10) kommen die Charaktere zu ihren eigenen. Viele Zauberkünstler erhalten zu Beginn dieser Stufe Zugang zu Zaubern der 3. Stufe, die eine neue Schwelle magischer Macht mit Zaubersprüchen wie Feuerball und Blitz überschreiten. In dieser Stufe erhalten viele Klassen, die Waffen verwenden, die Fähigkeit, in einer Runde mehrere Angriffe auszuführen. Diese
Charaktere sind wichtig geworden, da sie Gefahren ausgesetzt sind, die Städte und Königreiche bedrohen.
In der dritten Reihe (Level 11–16) haben die Charaktere eine Macht erreicht, die sie über die normale Bevölkerung stellt und sie sogar unter Abenteurern besonders macht. Auf der 11. Stufe erhalten viele Zauberkünstler Zugang zu Zaubern der 6. Stufe, von denen einige Effekte erzeugen, die zuvor für Spieler nicht möglich waren. Andere Charaktere erhalten Funktionen, mit denen sie mehr Angriffe ausführen oder mit diesen Angriffen eindrucksvollere Aktionen ausführen können. Diese mächtigen Abenteurer sehen sich oft Bedrohungen für ganze Regionen und Kontinente gegenüber.

In der vierten Reihe (Level 17–20) erreichen die Charaktere den Höhepunkt ihrer Klassenmerkmale und werden selbst zu heroischen (oder bösartigen) Archetypen. Das Schicksal der Welt oder sogar die grundlegende Ordnung des Multiversums könnte bei ihren Abenteuern in der Balance stehen.
\newline
\header{Entwicklungsstufen}\newline
\begin{minipage}{.2\linewidth}
	\begin{dndtable}
	   	\textbf{XP} \\
			0 \newline
			300 \newline
			900 \newline
			2.700 \\
			6.500 \newline
			14.000 \newline
			23.000 \newline
			34.000 \newline
			48.000 \newline
			64.000 \\
			85.000 \newline
			100.000 \newline
			120.000 \newline
			140.000 \newline
			165.000 \newline
			195.000 \\
			225.000 \newline
			265.000 \newline
			305.000 \newline
			355.000
	\end{dndtable}
\end{minipage}
\begin{minipage}{.2\linewidth}
	\begin{dndtable}
	   	\textbf{Level} \\
			1 \newline
			2 \newline
			3 \newline
			4 \\
			5 \newline
			6 \newline
			7 \newline
			8 \newline
			9 \newline
			10 \\
			11 \newline
			12 \newline
			13 \newline
			14 \newline
			15 \newline
			16 \\
			17 \newline
			18 \newline
			19 \newline
			20
	\end{dndtable}
\end{minipage}
\begin{minipage}{.2\linewidth}
	\begin{dndtable}
	   	\textbf{Level} \\
		\	+2 \newline
		\	+2 \newline
		\	+2 \newline
		\	+2 \\
		\	+3 \newline
		\	+3 \newline
		\	+3 \newline
		\	+3 \newline
		\	+4 \newline
		\	+4 \\
		\	+4 \newline
		\	+4 \newline
		\	+5 \newline
		\	+5 \newline
		\	+5 \newline
		\	+5 \\
		\	+6 \newline
		\	+6 \newline
		\	+6 \newline
		\	+6
	\end{dndtable}
\end{minipage}
